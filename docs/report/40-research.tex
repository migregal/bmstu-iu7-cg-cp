\chapter{Экспериментальная часть}

В данном разделе будет поставлен эксперимент, в котором будут сравнены временные характеристики работы реализованного программного обеспечения в различных конфигурациях.

\section{Цель эксперимента}

Целью эксперимента является проверка правильности выполнения поставленной задачи, оценка эффективности при многопоточной реализации синтеза изображения, а так же - сравнение эффективности работы многопоточной реализации при различном количестве потоков.

\section{Апробация}

На Рисунках \ref{img:probe1} -- \ref{img:probe5} представлены результаты синтеза изображения с различными параметрами освещенности сцены. Можно заметить, что сцены работают корректно.

\imgw{probe1}{ht!}{\textwidth}{Визуализация сцены в обычном режиме}

\imgw{probe2}{ht!}{\textwidth}{Визуализация сцены только с точечным источником света максимальной интенсивности}

\imgw{probe3}{ht!}{\textwidth}{Визуализация сцены только с направленным источником света максимальной интенсивности}

\imgw{probe4}{ht!}{\textwidth}{Визуализация сцены только с рассеянным светом максимальной интенсивности}

\imgw{probe5}{ht!}{\textwidth}{Визуализация сцены без источников освещения}

\clearpage
На Рисунках \ref{img:probe6} -- \ref{img:probe7} показана одна и та же фрактальная поверхность при разном положении источника. Смена положения источника освещения работает корректно.

\imgw{probe6}{ht!}{\textwidth}{Визуализация сцены}

\imgw{probe7}{ht!}{\textwidth}{Визуализация сцены со смещенным направленным источником}

\section{Технические характеристики}

Технические характеристики устройства, на котором выполнялось исследование:

\begin{itemize}
	\item процессор: Intel Core™ i5-8250U \cite{i5} CPU @ 1.60GHz;
	\item память: 32 GiB;
	\item операционная система: Manjaro \cite{manjaro} Linux \cite{linux} 21.1.4 64-bit.
\end{itemize}

Исследование проводилось на ноутбуке, включенном в сеть электропитания. Во время тестирования ноутбук был нагружен только встроенными приложениями окружения рабочего стола, окружением рабочего стола, а также непосредственно системой тестирования.

\section{Описание эксперимента}

Была реализована функция параллельного синтеза сцены. Для этого была использована библиотека \texttt{OpenMP} \cite{omp}, директива препроцессора \texttt{\#pragma omp parallel for}. Данная директива препроцессора преобразует код для выполнения итераций цикла параллельно. 

В данном эксперименте полученное изображение разбивалось на массивы рядов, во избежание проблем с разделяемой памятью (класс \texttt{QImage} \cite{qimage}, предоставляемый фреймворком \texttt{QT} \cite{qt}, не может использоваться для параллельных записи и чтения), а по окончании собирались обратно.

В рамках данного эксперимента будет производиться оценка влияния размерности изображения и количества потоков на время работы алгоритма. Для этого будем синтезировать квадратные изображения с размерностями равными [300, 400, 500, \dots, 1000] элементов. Количество потоков будет задано равным [1, 2, 4, 8, 16] штук.
 
Для снижения погрешности измерений будем усреднять получаемые значения. Для этого каждое из измерений будет проводиться $N = 100$ раз, после чего будет вычисляться среднее арифметическое значение измеряемой величины.

\section{Результат эксперимента}

В таблице \ref{tbl:time} приведены экспериментально полученные значения временных характеристик работы алгоритма в зависимости от размерности синтезируемого изображения и количества потоков.

\begin{table}[ht]
	\small
	\begin{center}
		\caption{Замеры времени для изображений с различными размерностями}
		\label{tbl:time}
		\begin{tabular}{|c|c|c|c|c|c|}
			\hline
			\bfseries Размерность & \multicolumn{5}{c|}{\bfseries Количество потоков, шт.} \\ \cline{2-6}
			\bfseries изображения, пикс. & \bfseries 1 & \bfseries 2 & \bfseries 4 & \bfseries 8 & \bfseries 16
			\csvreader{inc/csv/time.csv}{}
			{\\\hline \csvcoli&\csvcolii&\csvcoliii&\csvcoliv&\csvcolv&\csvcolvi}
			\\\hline
		\end{tabular}
	\end{center}
\end{table}

Согласно данным, приведенным в таблице \ref{tbl:time}, время синтеза изображения зависит от размерности данного изображения как $O(m n)$ или $O(n^2)$ т.к. в данном эксперименте синтезируются только квадратные изображения.

На рисунке \ref{plt:time} приведены графики зависимости времени синтеза изображения от размерности синтезируемого изображения для различного числа потоков, использующихся в алгоритме.

\begin{figure}[ht]
	\centering
	\begin{tikzpicture}
		\begin{axis}[
			axis lines=left,
			xlabel={Размерность синтезируемого изображения},
			ylabel={Время, мс},
			legend pos=north west,
			ymajorgrids=true
		]
			\addplot table[x=size,y=1,col sep=comma] {inc/csv/time.csv};
			\addplot table[x=size,y=2,col sep=comma] {inc/csv/time.csv};
			\addplot table[x=size,y=4,col sep=comma] {inc/csv/time.csv};
			\addplot table[x=size,y=8,col sep=comma] {inc/csv/time.csv};
			\addplot table[x=size,y=16,col sep=comma] {inc/csv/time.csv};
			\legend{1,2,4,8,16}
		\end{axis}
	\end{tikzpicture}
	\captionsetup{justification=centering}
	\caption{Сравнение времени работы алгоритмов}
	\label{plt:time}
\end{figure}

Из рисунка \ref{plt:time} следует, что наиболее эффективным реализованный алгоритм синтеза изображения становится при приближении числа потоков к количеству логических ядер используемой ЭВМ (8 для использованной в ходе эксперимента). Так, при синтезе изображений с размерностью равной $1000$ пикселей, алгоритм использующий 8 потоков оказался в $5.25$ раз эффективнее однопоточного алгоритма по времени.

Стоит отметить, что многопоточная реализация алгоритма оказалась  более эффективной при всех рассматриваемых размерностях изображения. Связано это с большими размерностями синтезируемых изображений (от 300 до 1000 пикселей). Для изображения малых размерностей, однопоточный алгоритм окажется более эффективным в связи с отсутствием дополнительных затрат по времени и памяти, требуемых для реализации многопоточности (создание потоков, совместный лоступ к ресурсам).

\section{Вывод}

В данном разделе было произведено экспериментально сравнение временных характеристик реализованного программного обеспечения.

Время работы алгоритма имеет квадратичную зависимость от размерности синтезируемого изображения.

Наиболее эффективной по времени оказалась многопоточная реализация с числом потоков равным 8 (число логических ядер ЭВМ использованной в ходе эксперимента).