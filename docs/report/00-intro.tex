\chapter*{Введение}
\addcontentsline{toc}{chapter}{Введение}

Физические тела, окружающие нас, обладают различными оптическими свойствами. Они, к примеру, могут отражать или пропускать световые лучи, также они могут отбрасывать тень. Эти и другие свойства нужно уметь наглядно показывать при помощи электронно-вычислительных машин.
Этим и занимается компьютерная графика.

\textit{Компьютерная графика} -- представляет собой совокупность методов и способов преобразования информации в графическое представление при помощи ЭВМ.
Без компьютерной графики не обходится ни одна современная программа. В течении нескольких десятилетий компьютерная графика прошла долгий путь, начиная с базовых
алгоритмов, таких как вычерчивание линий и отрезков, до построения виртуальной реальности.

Целью данного курсового проекта является разработка ПО, визуализирующего трехмерные
фрактальные поверхности.

Для достижения данной цели необходимо решить следующие задачи:

\begin{itemize}
    \item описать структуру синтезируемой трехмерной сцены;
    \item описать существующие алгоритмы построения реалистичных изображений;
    \item выбрать и обосновать выбор реализуемых алгоритмов;
    \item привести схемы реализуемых алгоритмов;
    \item определить требования к программному обеспечению;
	\item описать использующиеся структуры данных;
	\item описать структуру разрабатываемого ПО;
	\item определить средства программной реализации;
    \item реализовать соответствующее ПО;
	\item провести экспериментальные замеры временных характеристик разработанного ПО.
\end{itemize}
