\chapter{Технологическая часть}

В данном разделе будут представлены средства разработки программного обеспечения, детали реализации и процесс сборки разрабатываемого программного обеспечения.

\section{Выбор средств реализации}

В качестве языка программирования для разработки программного обеспечения был выбран язык \texttt{C++} \cite{cpp}. Данный выбор обусловлен тем, что данный язык предоставляет весь функционал требуемый для решения поставленной задачи.

Для создания пользовательского интерфейса ПО был использован фреймворк \texttt{QT} \cite{qt}. Данный фреймворк содержит в себе объекты, позволяющие напрямую работать с пикселями изображения, а  так же возможности создания интерактивных пользовательских интерфейсов, что позволит в интерактивном режиме управлять изображением.

В процессе работы был использован инструмент \texttt{Clang} \cite{clang}, позволяющий форматировать исходные коды, а так же в процессе их написания обнаруживать наличие синтаксических ошибок. Так же, данный инструмент предоставляет возможность статического анализа кода \cite{clang_static}. 

Кроме того, использовался инструмент \texttt{Valgrind} \cite{valgrind}, позволяющий отслеживать утечки памяти в ходе работы программного обеспечения.

Для сборки программного обеспечения использовались инструменты \texttt{make} \cite{make} и \texttt{CMake} \cite{cmake}.

В качестве среды разработки был выбран текстовый редактор \texttt{Visual Studio Code} \cite{vscode}, поддерживающий возможность установки плагинов \cite{vscode_plugins}, в том числе для работы с \texttt{C++} и \texttt{CMake}.

\clearpage
\section{Интерфейс ПО}

Интерфейс реализуемого ПО представлен на Рисунках \ref{img:ui1} -- \ref{img:ui3}.

\imgw{ui1}{ht!}{0.6\textwidth}{Интерфейс программы - группы настроек построения поверхности}

На Рисунке \ref{img:ui1} представлен интерфейс настройки параметров построения фрактальной поверхности, включающий в себя выбор поверхности для построения, специфичных для конкретной поверхности параметров, интервала построения по осям X, Y и Z, характеристик материала согласно ТЗ, а так же - признака использования сглаживания в процессе синтеза сцены.
\clearpage

\imgw{ui2}{ht!}{0.6\textwidth}{Интерфейс программы - группы настроек камеры}

На Рисунке \ref{img:ui2} представлен интерфейс настройки параметров камеры (точки наблюдения), включающий в себя задание параметров и выполнение в соответствии с ними поворота камеры относительно вектора (X, Y, Z) на угол Угол, задание параметров и выполнение в соответствии с ними перемещения камеры относительно текущего положения камеры на вектор (X, Y, Z).

\clearpage
\imgw{ui3}{ht!}{0.6\textwidth}{Интерфейс программы - группы настроек освещения сцены}

На Рисунке \ref{img:ui3} представлен интерфейс настройки параметров освещения синтезируемой сцены, включающий в себя задание параметров различных источников освещения в соответствии с их описанием, приведенным ранее, а именно:
\begin{itemize}
    \item Интенсивности для внешнего (рассеянного) света;
    \item Интенсивности, а так же вектора направления (X, Y, Z) направленного источника света;
    \item Интенсивности, а так же положения (X, Y, Z) точечного источника света;
\end{itemize}

\section{Реализация алгоритмов}

В листинге \ref{lst:raycast} представлена реализация алгоритма \texttt{Ray Marching}'а. В листинге \ref{lst:lightning} представлена реализация алгоритма расчета освещенности.

\listingfile{Fractal.cpp}{raycast}{C++}{Реализация алгоритма Ray Marching}{linerange={1-35}}
\clearpage

\listingfile{Lightning.cpp}{lightning}{C++}{Реализация алгоритма расчета освещенности}{}
\clearpage

\section{Описание процесса сборки приложения}

Для сборки программного обеспечения использовались инструменты \texttt{make} \cite{make} и \texttt{CMake} \cite{cmake}.

Действия, необходимые для сборки проекта приведены приведены в листинге \ref{lst:build}:

\listingfile{build.sh}{build}{bash}{Сборка реализуемого программного обеспечения}{}

В результате выполнения данных действий будет скомпилирован исполняемый файл \texttt{bmstu\_iu7\_cg\_course}.

\section{Вывод}

В данном разделе были представлены средства разработки программного обеспечения, детали реализации и процесс сборки разрабатываемого программного обеспечения.