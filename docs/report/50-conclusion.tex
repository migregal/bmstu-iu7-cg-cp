\chapter*{Заключение}
\addcontentsline{toc}{chapter}{Заключение} 

В ходе курсового проекта было разработано программное обеспечение, предоставляющее возможность визуализации трехмерных фрактальных поверхностей. Разработанное программное обеспечение предоставляет функционал для изменения интервалов построения поверхностей, задания цвета и свойств материала поверхности, а так же задания и изменения в процессе работы положения точки наблюдения и источников света по их характеристикам (положению, интенсивности) в интерактивном режиме. В процессе выполнения данной работы были выполнены следующие задачи:

\begin{itemize}
    \item описана структура синтезируемой трехмерной сцены;
    \item описаны существующие алгоритмы построения реалистичных изображений;
    \item были выбраны реализуемые алгоритмы;
    \item приведены схемы реализуемых алгоритмов;
    \item определены требования к программному обеспечению;
	\item описаны использующиеся структуры данных;
	\item описана структура разрабатываемого ПО;
	\item определены средства программной реализации;
	\item реализовано соответствующее ПО;
	\item проведены экспериментальные замеры временных характеристик разработанного ПО.
\end{itemize}
