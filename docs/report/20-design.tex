\chapter{Конструкторская часть}

В данном разделе будут рассмотрены требования к программному обеспечению, а также схемы алгоритмов, выбранных для решения поставленной задачи. Так же, будут описаны пользовательские структуры данных и приведена структура реализуемого программного обеспечения.

\section{Требования к программному обеспечению}

ПО должно предоставлять доступ к следующему функционалу:

\begin{itemize}
    \item выбор фрактальной поверхности из предложенного списка;
    \item задание интервала построения для выбранной фрактальной поверхности по ося \texttt{X}, \texttt{Y}, \texttt{Z};
    \item задание цвета и свойств поверхности;
    \item поворот и перемещение камеры (точки наблюдения);
    \item изменение положения, интенсивности и ориентации источника света.
\end{itemize}

К ПО предъявляются следующие требования:

\begin{itemize}
    \item время отклика программы не должно превышать 30 секунд для корректной работы в интерактивном режиме;
    \item программа должна корректно реагировать на люые действия пользователя.
\end{itemize}

\section{Разработка алгоритмов}

В данном проекте алгоритм \texttt{Ray Marching} будет применяться к фрактальным поверхностям.

Эффективность процедуры определения пересечений луча с фрактальной поверхностью оказывает самое большое влияние на эффективность всего алгоритма. Чтобы избавиться от излишнего поиска пересечений, рассматривают пересечение луча с объемной оболочкой рассматриваемого объекта. В данном случае под оболочкой понимается некоторый простой объект, внутрь которого можно поместить рассматриваемый объект, к примеру параллелепипед или сфера.

На рис. \ref{img:shell} приведены примеры сферической и прямоугольной оболочки.

\imgs{shell}{h!}{0.9}{Сферическая и прямоугольная оболочки}

Если такого пересечения нет, то и, соответственно, пересечения луча и
самого рассматриваемого объекта нет, и наоборот, если найдено пересечение, то, возможно, существует пересечение луча и рассматриваемого объекта.

Для повышения эффективности процедуры определения пересечений луча с прямоугольной оболочкой, данную оболочку часто заменяют на \texttt{AABB}, что позволяет упростить вычисления при незначительной потере точности.

\subsection{Пересечение луча и сферы}

Уравнение луча представлено ниже:

\begin{equation}
	P = O + t\overrightarrow{D}, t \geq 0,
	\label{eq:ref5}
\end{equation}
где $\overrightarrow{D}$ -- направление луча.

Сфера — это множество точек $P$, лежащих на постоянном расстоянии $r$ от фиксированной точки $C$. Тогда можно записать уравнение, удовлетворяющее этому условию:

\begin{equation}
	distance(P,C) = r
	\label{eq:ref6}
\end{equation}

Запишем расстояние (\ref{eq:ref6}) между P и C как длину вектора из P в C.

\begin{equation}
	|P-C|=r
\end{equation}

Заменим на скалярное произведение вектора на себя:

\begin{equation}
	\sqrt{\langle P - C\rangle, \langle P - C\rangle} = r
\end{equation}

Избавимся от корня:

\begin{equation}
	\langle P - C\rangle, \langle P - C\rangle = r^2
	\label{eq:ref7}
\end{equation}

В итоге есть два уравнения - уравнение луча и сферы. Найдем пересечение луча со сферой. Для этого подставим (\ref{eq:ref5}) в (\ref{eq:ref7})

\begin{equation}
	\langle O + t\overrightarrow{D} - C \rangle, \langle O + t\overrightarrow{D} - C\rangle = r^2
\end{equation}

Разложим скалярное произведение и преобразуем его. В результате получим:

\begin{equation}
	t^2 \langle \overrightarrow{D}, \overrightarrow{D} \rangle + 2t \langle \overrightarrow{OC}, \overrightarrow{D} \rangle + \langle \overrightarrow{OC}, \overrightarrow{OC} \rangle -r^2 = 0
	\label{eq:ref8}
\end{equation}

Представленное квадратное уравнение (\ref{eq:ref8}) имеет несколько возможнных случаев решения.
Если у уравнения одно решение, это обозначает, что луч касается сферы.
Два решения обозначают, то что луч входит в сферу и выходит из неё.
И если нет решений, значит, луч не пересекается со сферой.

\subsection{Пересечение луча и AABB}

Пример \texttt{AABB} приведен на Рисунке \ref{img:aabb}.

\imgs{aabb}{ht!}{0.7}{Пример \texttt{AABB}}

Уравнение \eqref{eq:ref5_1} эквивалентно уравнению \eqref{eq:ref5}

\begin{equation}
	{\begin{cases}
			x(t) = x_O + t x_D \\
			y(t) = y_O + t y_D \\
			z(t) = z_O + t z_D
			\label{eq:ref5_1}
		\end{cases}}
\end{equation}

Из определения \texttt{AABB} следует, что каждая из граней такой оболочки задается некоторым константным выражением, в связи с чем из уравнения \eqref{eq:ref5_1} можно выразить значение параметра $t$, соответствующее точке пересечения луча и соответствующей плоскости:

\begin{equation}
	t = \frac{P_x - x_O}{x_D}
	\label{eq:t_aabb_1}
\end{equation}

\begin{equation}
    t = \frac{P_y - y_O}{y_D}
    \label{eq:t_aabb_2}
\end{equation}

\begin{equation}
    t = \frac{P_z - z_O}{z_D}
	\label{eq:t_aabb_3}
\end{equation}

Теперь найдем значения параметра $t$ для каждой из пар параллельных плоскостей, содержащих грани рассматриваемой \texttt{AABB}, воспользовавшись уравнениями \eqref{eq:t_aabb_1} -- \eqref{eq:t_aabb_3}. Конкретное уравнение выбирается в зависимости от положения рассматриваемой пары плоскостей в пространстве.

Обозначим вычисленные значения как $T_{near}$ и $T_{far}$, определяющие значение параметра $t$ в точке пересечения с ближайшей и дальней плоскостями из рассматриваемой пары соответственно.

В результате, может быть сделан следующий вывод:

\begin{equation}
	{\begin{cases}
			\texttt{AABB}\textup{ расположен вне зоны видимости},& T_{far} < 0 \\
			\textup{луч не пересекает }\texttt{AABB},& T_{far} < T_{min} \\
			\textup{обнаружено пересечение с }\texttt{AABB},& \textup{иначе}
			\label{eq:ref5_3}
		\end{cases}}
\end{equation}

\subsection{Описание алгоритма Ray Marching-а}

Суть алгоритма состоит в следующем: Из некоторой точки пространства, называемой виртуальным глазом, или камерой, через каждый пиксель изображения испускается луч и находится точка пересечения с ограничивающей оболочкой фрактальной поверхности (\ref{eq:ref8} и \ref{eq:ref5_3}). При обнаружении точки пересечения, вычисляется точка пересечения луча с фрактальной поверхностью. Далее из найденной точки пересечения испускаются лучи до каждого источника освещения. Если данные лучи пересекают другие объекты сцены, значит точка пересечения находится в тени относительно рассматриваемого источника освещения и освещать ее не нужно. Освещение со всех видимых источников света складываются  (по интенсивности).

На Рисунке \ref{img:ray_marching_diagram} представлена схема синтеза изображения с применением данного алгоритма.

\imgs{ray_marching_diagram}{ht!}{0.6}{Схема алгоритма синтеза изображения с применением алгоритма \texttt{ray marching}-а}

\section{Описание используемых типов и структур данных}

В данной работе используются следующие типы и структуры данных:

\begin{itemize}
    \item источник света -- задается расположением, направленностью и интенсивностью света;
    \item математические абстракции:
    \begin{itemize}
        \item точка –- хранит координаты x, y, z;
        \item вектор -– хранит направление по x, y, z;
    \end{itemize}
	\item цвет -- хранит три составляющие \texttt{RGB} модели цвета.
\end{itemize}

\section{Описание структуры программного обеспечения}

На Рисунке \ref{img:uml} представлена диаграмма классов реализуемого программного обеспечения.

\imgs{uml}{ht!}{0.85}{Схема алгоритма \texttt{ray marching}-а}


\section {Описание оптимизаций временных характеристик}

Из схемы алгоритма, представленной на Рисунке \ref{img:ray_marching_diagram} следует, что каждый пиксель экрана обрабатывается независимо от остальных, в связи с чем данный алгоритм можно оптимизировать, распараллелив вычисления.

В связи с аппаратными ограничениями \cite{amdal}, обработка каждого отдельного пикселя в отдельном потоке приведет лишь к снижению производительности разрабатываемого ПО. В связи с этим, следует разбить рассматриваемые пиксели на группы по "строкам" или "столбцам" и обрабатывать каждую группу отдельным потоком.

\section{Вывод}

В данном разделе были представлены требования к разрабатываемому программному обеспечению и разработана схема разрабатываемого алгоритма. Так же, были описаны пользовательские структуры данных и приведена структура реализуемого программного обеспечения.